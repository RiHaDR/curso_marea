\documentclass[]{book}
\usepackage{lmodern}
\usepackage{amssymb,amsmath}
\usepackage{ifxetex,ifluatex}
\usepackage{fixltx2e} % provides \textsubscript
\ifnum 0\ifxetex 1\fi\ifluatex 1\fi=0 % if pdftex
  \usepackage[T1]{fontenc}
  \usepackage[utf8]{inputenc}
\else % if luatex or xelatex
  \ifxetex
    \usepackage{mathspec}
  \else
    \usepackage{fontspec}
  \fi
  \defaultfontfeatures{Ligatures=TeX,Scale=MatchLowercase}
\fi
% use upquote if available, for straight quotes in verbatim environments
\IfFileExists{upquote.sty}{\usepackage{upquote}}{}
% use microtype if available
\IfFileExists{microtype.sty}{%
\usepackage{microtype}
\UseMicrotypeSet[protrusion]{basicmath} % disable protrusion for tt fonts
}{}
\usepackage[margin=1in]{geometry}
\usepackage{hyperref}
\hypersetup{unicode=true,
            pdftitle={Costeo y Evaluación de Reservas Marinas},
            pdfauthor={Juan Carlos Villaseñor-Derbez},
            pdfborder={0 0 0},
            breaklinks=true}
\urlstyle{same}  % don't use monospace font for urls
\usepackage{natbib}
\bibliographystyle{apalike}
\usepackage{longtable,booktabs}
\usepackage{graphicx,grffile}
\makeatletter
\def\maxwidth{\ifdim\Gin@nat@width>\linewidth\linewidth\else\Gin@nat@width\fi}
\def\maxheight{\ifdim\Gin@nat@height>\textheight\textheight\else\Gin@nat@height\fi}
\makeatother
% Scale images if necessary, so that they will not overflow the page
% margins by default, and it is still possible to overwrite the defaults
% using explicit options in \includegraphics[width, height, ...]{}
\setkeys{Gin}{width=\maxwidth,height=\maxheight,keepaspectratio}
\IfFileExists{parskip.sty}{%
\usepackage{parskip}
}{% else
\setlength{\parindent}{0pt}
\setlength{\parskip}{6pt plus 2pt minus 1pt}
}
\setlength{\emergencystretch}{3em}  % prevent overfull lines
\providecommand{\tightlist}{%
  \setlength{\itemsep}{0pt}\setlength{\parskip}{0pt}}
\setcounter{secnumdepth}{5}
% Redefines (sub)paragraphs to behave more like sections
\ifx\paragraph\undefined\else
\let\oldparagraph\paragraph
\renewcommand{\paragraph}[1]{\oldparagraph{#1}\mbox{}}
\fi
\ifx\subparagraph\undefined\else
\let\oldsubparagraph\subparagraph
\renewcommand{\subparagraph}[1]{\oldsubparagraph{#1}\mbox{}}
\fi

%%% Use protect on footnotes to avoid problems with footnotes in titles
\let\rmarkdownfootnote\footnote%
\def\footnote{\protect\rmarkdownfootnote}

%%% Change title format to be more compact
\usepackage{titling}

% Create subtitle command for use in maketitle
\newcommand{\subtitle}[1]{
  \posttitle{
    \begin{center}\large#1\end{center}
    }
}

\setlength{\droptitle}{-2em}

  \title{Costeo y Evaluación de Reservas Marinas}
    \pretitle{\vspace{\droptitle}\centering\huge}
  \posttitle{\par}
    \author{Juan Carlos Villaseñor-Derbez}
    \preauthor{\centering\large\emph}
  \postauthor{\par}
      \predate{\centering\large\emph}
  \postdate{\par}
    \date{Bren School of Environmental Science \& Management, University of
California Santa Barbara}

\usepackage{booktabs}
\usepackage{amsthm}
\makeatletter
\def\thm@space@setup{%
  \thm@preskip=8pt plus 2pt minus 4pt
  \thm@postskip=\thm@preskip
}
\makeatother

\begin{document}
\maketitle

{
\setcounter{tocdepth}{1}
\tableofcontents
}
\hypertarget{antes-de-empezar}{%
\chapter*{Antes de empezar}\label{antes-de-empezar}}
\addcontentsline{toc}{chapter}{Antes de empezar}

Este manual es la segunda iteración de los esfuerzos por impulsar el uso
de metodologías estandarizadas para la evaluación de reservas marinas.
Trabajos anteriores incluyen el manual generalizado de evaluación de
reservas marinas en México \citep{villaseorderbez_2017} y la publicación
arbitrada que presenta a
\href{https://turfeffect.shinyapps.io/marea/}{MAREA} como una
herramienta amigable y gratuita \citep{villasenorderbez_2018}. Esta
versión del manual pretende incorporar partes de ambos trabajos, pero
también incluye una serie de ejercicios prácticos para el uso de
\href{https://turfeffect.shinyapps.io/marea/}{MAREA} y la nueva
\href{https://turfeffect.shinyapps.io/AppCosteo/}{App de Costeo de
Reservas}. Además, el manual está públicamente disponible en
\href{https://jcvdav.github.io/curso_marea/}{internet}, donde el lector
puede descargar el manual como PDF o EPUB para Kindle.

\hypertarget{requisitos}{%
\section{Requisitos}\label{requisitos}}

\href{https://turfeffect.shinyapps.io/marea/}{MAREA} y la nueva
\href{https://turfeffect.shinyapps.io/AppCosteo/}{App de Costeo de
Reservas} son aplicaciones web, y para poder utilizarlas es necesario
tener un explorador de internet y una conexión estable. Aunque no
siempre tenemos acceso a internet, este formato nos evita problemas de
compatibilidad entre diferentes sistemas operativos. Si tienes un
explorador de internet y una conexión estable, puedes usar estas Apps.

Si participaste en uno de los cursos presenciales, el USB que recibise
contiene este manual como PDF y EPUB además de los
\href{https://github.com/jcvdav/curso_marea/materiales/datos}{datos
sintéticos} para los ejercicios prácticos y las
\href{https://github.com/jcvdav/curso_marea/materiales/diapositivas}{diapositivas
del curso}. Puedes distribuir libremente estos materiales, o
descargarlos desde el
\href{https://github.com/jcvdav/curso_marea}{repositorio de GitHub}. La
versión en línea siempre será la más actualizada.

\hypertarget{introduccion}{%
\chapter{Introducción}\label{introduccion}}

\hypertarget{antecedentes-en-la-evaluacion-de-reservas}{%
\chapter{Antecedentes en la evaluación de
reservas}\label{antecedentes-en-la-evaluacion-de-reservas}}

\hypertarget{dentro-vs.fuera}{%
\section{Dentro vs.~Fuera}\label{dentro-vs.fuera}}

\hypertarget{antes-vs.despues}{%
\section{Antes vs.~Después}\label{antes-vs.despues}}

\hypertarget{dentro-vs.fuera---antes-vs.despues}{%
\section{Dentro vs.~Fuera - Antes
vs.~Después}\label{dentro-vs.fuera---antes-vs.despues}}

\hypertarget{dentro-vs.fuera---antes-vs.despues-multiple}{%
\section{Dentro vs.~Fuera - Antes vs.~Después
multiple}\label{dentro-vs.fuera---antes-vs.despues-multiple}}

\hypertarget{evaluacion-de-reservas}{%
\chapter{Evaluación de reservas}\label{evaluacion-de-reservas}}

\hypertarget{objetivos-e-indicadores}{%
\section{Objetivos e indicadores}\label{objetivos-e-indicadores}}

\hypertarget{analisis-de-inferencia-de-causalidad}{%
\section{Análisis de inferencia de
causalidad}\label{analisis-de-inferencia-de-causalidad}}

\hypertarget{introduccion-a-marea}{%
\chapter{Introducción a MAREA}\label{introduccion-a-marea}}

\hypertarget{tipos-y-formatos-de-datos}{%
\section{Tipos y formatos de datos}\label{tipos-y-formatos-de-datos}}

\hypertarget{capacidades-y-limitaciones}{%
\section{Capacidades y limitaciones}\label{capacidades-y-limitaciones}}

\hypertarget{evaluacion-de-reservas-en-6-etapas}{%
\section{Evaluación de reservas en 6
etapas}\label{evaluacion-de-reservas-en-6-etapas}}

\hypertarget{interpretacion-de-resultados}{%
\section{Interpretación de
resultados}\label{interpretacion-de-resultados}}

\hypertarget{uso-de-marea}{%
\chapter{Uso de MAREA}\label{uso-de-marea}}

\hypertarget{evaluacion-de-indicadores-biologicos-para-1-reserva}{%
\section{Evaluación de indicadores biológicos para 1
reserva}\label{evaluacion-de-indicadores-biologicos-para-1-reserva}}

\hypertarget{evaluacion-de-indicadores-biologicos-y-especie-objetivo-para-1-reserva}{%
\section{Evaluación de indicadores biológicos y especie objetivo para 1
reserva}\label{evaluacion-de-indicadores-biologicos-y-especie-objetivo-para-1-reserva}}

\hypertarget{evaluacion-de-todos-los-indicadores-para-1-reserva}{%
\section{Evaluación de todos los indicadores para 1
reserva}\label{evaluacion-de-todos-los-indicadores-para-1-reserva}}

\hypertarget{evaluacion-de-todos-los-indicadores-para-varias-reservas-simultaneamente}{%
\section{Evaluación de todos los indicadores para varias reservas,
simultáneamente}\label{evaluacion-de-todos-los-indicadores-para-varias-reservas-simultaneamente}}

\hypertarget{errores-comunes-y-solucion-de-problemas}{%
\chapter{Errores comunes y solución de
problemas}\label{errores-comunes-y-solucion-de-problemas}}

\hypertarget{especie-indicador-no-tiene-diseno-baci}{%
\section{Especie / Indicador no tiene diseño
BACI}\label{especie-indicador-no-tiene-diseno-baci}}

\hypertarget{diferentes-especies-en-bases-biologicas-vs-pesca}{%
\section{Diferentes especies en bases biológicas vs
pesca}\label{diferentes-especies-en-bases-biologicas-vs-pesca}}

\bibliography{references.bib}


\end{document}
